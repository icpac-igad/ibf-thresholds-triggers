
\chapter{Thresholds and triggers for Anticipatory Action in Karamoja Region}


\section{Introduction} 

Documentation on analysis for verification of Seasonal drought forecast for Anticipatory action in Karamoja region, thresholds and triggers 

"A trigger is a forecast that is issued at a certain lead time which exceeds both the danger level and the probability threshold leading to the initiation of predefined actions."

In Karamoja, the danger level for drought defined as thresholds(Moderatem,Extreme, Severe) is predefined by analysis on the frequency of long-term SPI values observed in Karamoja as shown in the figure \ref{fig:a} and then stackholder consultation by comparing it with the drought disaster impact faced by the region \ref{tab:incpacks}. The current analysis explores the long term monthly SPI variability in Karamoja with verification on forecast quality and determines the trigger for each threshold in terms of probability for defined danger levels for SPI products namely, SPI3-MAM, SPI4-JJAS and SPI6 as shown in the figure \ref{fig:b}.

The end product is a trigger value(Emprical probability of Ensemble forecasts exceeding the given threshold ranges) reflected in terms of False Alarm Ratio (output shown in figure \ref{fig:far}) and Hit Rate (output shown in figure \ref{fig:hr})with respect to different lead time, threshold levels and region/district. The output of empirical probabilities found for the period 1981-2023 for the Karamoja region is shown in figures \ref{fig:ep1} to \ref{fig:ep9}

\begin{figure*}
	\center
	{\includegraphics[width=0.7\columnwidth]{img2}}
	%\raisebox{3in}[0in][0in]{\color{red}
	%\makebox[\textwidth][c]{\Huge Text over Image}}
	\caption{Threshold selection for Karamoja region based on long term observed SPI values}\label{fig:a}
\end{figure*}


\begin{figure*}
\center
{\includegraphics[width=0.7\columnwidth]{img1}}
%\raisebox{3in}[0in][0in]{\color{red}
%\makebox[\textwidth][c]{\Huge Text over Image}}
\caption{Lead time and SPI products available for verification for Karamjoa region based on Growing and rainy season}\label{fig:b}
\end{figure*}


\begin{table}[!ht]
	\centering
	\caption[SPI threshold for Karamoja ]{SPI threshold for Karamoja.}
	\label{tab:incpacks}
	\begin{tabular}{p{0.2\columnwidth}p{0.2\columnwidth}p{0.2\columnwidth}p{0.2\columnwidth}}
		\toprule
		&\textbf{Moderate} &\textbf{Extreme} &\textbf{Severe}\\
		\midrule
		%accessibility & tagged & generates the document structure and tagging \\
	MAM &-0.03 to -0.55 &-0.56 to -0.98 &-0.99 \\
	JJAS &-0.01 to -0.40 &-0.41 to -0.98 &-0.99 \\
	MAMJJAS or SPI 6 &-0.02 to -0.37 &-0.38 to -1.00 &-1.01 \\
		\bottomrule
	\end{tabular}
\end{table}




The method closely follows this paper,

Nobre, Gabriela Guimarães, et al. "Forecasting, thresholds, and triggers: Towards developing a Forecast-based Financing system for droughts in Mozambique." Climate Services 30 (2023): 100344. 

Apart from text in the paper,supplementary of the paper has a number of different plots, this method follows to recreate those plots and conclude at which triggers (probability values for each threshold) to choose based on the verification matrices. The plots can be grouped as follows and which can be taken as the steps described in the paper as well.  

\begin{enumerate}
	\item Observation threshold bar chart (thresholds are already determined based on workshop consultation, figure A and table 1) 
	\item Forecast probabilities for each SPI product and lead time 
	\item Verification metrices, HSS, BS, HKS for range of probabilities (testing of each year seasonal forecast with observations). This will be testing of each probability values for three thresholds from the plot in step 2. So this will be testing to find HSS, BS, HKS verification on possible combination of observation vs forecast. To plot range of probabilities and its verification score as below.
	\item FAR and HR on probabilities (triggers) filtered based on the quality of AUROC values and HSS, BS, HKS 
\end{enumerate}


\section{Analysis steps}

\begin{enumerate}

\item \textbf{Data Processing of SEAS5 and CHRIPS}, script for this step \url{https://github.com/icpac-igad/ibf-thresholds-triggers/blob/main/01_fcst_data_process.py}
\subitem CHRIPS monthly mean observation data, upscale 5km into 25 km 

\subitem SEAS5 forecast data, lead time, 1-6 months, downscale 100k into 25km 

\subitem SPI calculation on CHRIPS data 

\subitem Processing SEAS5 data 

\subitem Removing time steps 

\subitem Precipitation in m/s into mm/month 

\subitem SPI calculation on SEAS5 data, do SPI product and month lead time as shown in Figure 1 

\item \textbf{Emprical Probablity creation for ensemble forecast} script for this step \url{https://github.com/icpac-igad/ibf-thresholds-triggers/blob/main/02_prob_plot_q.py} and \url{https://github.com/icpac-igad/ibf-thresholds-triggers/blob/main/03_prob_csv_q.py}

\subitem Drought probabilities from SEAS5 ensemble output for each SPI product and lead time (for example, in case of MAM, probability plot of Figure2 for 11-MAM, 12-MAM, 1-MAM,2-MAM) 

\item \textbf{Forecast verification stats and plotting of outputs} \url{https://github.com/icpac-igad/ibf-thresholds-triggers/blob/main/04_metrics_csv_q.py} and \url{https://github.com/icpac-igad/ibf-thresholds-triggers/blob/main/05_table_plot.py}
\subitem Aligning SPI3 values of observation and forecast for each months shown in figure 1 (Months, 11,12,1,2 for SPI MAM) 
\subitem Contingency table creation from output of step 7 
\subitem Calculate forecast verification skill scores HSS, Bias Score, Hansen Kuipers Score 	
\end{enumerate}











\begin{figure*}
\noindent
\resizebox{\textwidth}{\textheight}
{\includegraphics{../output/tables/far_prob}}\hspace*{-\textwidth}
%\raisebox{3in}[0in][0in]{\color{red}
%\makebox[\textwidth][c]{\Huge Text over Image}}
\caption{Overview of False alarm ratio(\%) per SPI indicator, lead time of the forecasting information in months and region/District wise. The chosen trigger value is displayed within each tile  }\label{fig:far}
\end{figure*}

\begin{figure*}
	\noindent
	\resizebox{\textwidth}{\textheight}
	{\includegraphics{../output/tables/pod_prob}}\hspace*{-\textwidth}
	%\raisebox{3in}[0in][0in]{\color{red}
	%\makebox[\textwidth][c]{\Huge Text over Image}}
	\caption{Overview of Hit rate(\%) per SPI indicator, lead time of the forecasting information in months and region/District wise. The chosen trigger value is displayed within each tile}\label{fig:hr}
\end{figure*}


\begin{figure*}
	\noindent
	\resizebox{\textwidth}{\textheight}
	{\includegraphics{../output/prob_plot/mam_nov}}\hspace*{-\textwidth}
	%\raisebox{3in}[0in][0in]{\color{red}
	%\makebox[\textwidth][c]{\Huge Text over Image}}
	\caption{Time series of SPI 3 for the four districts based on the forecast batch of November. Lines in yellow, orange, and brown represent the probability of drought using threshold of mild, moderate, and severe categories, respectively.}\label{fig:ep1}
\end{figure*}

\begin{figure*}
	\noindent
	\resizebox{\textwidth}{\textheight}
	{\includegraphics{../output/prob_plot/mam_dec}}\hspace*{-\textwidth}
	%\raisebox{3in}[0in][0in]{\color{red}
	%\makebox[\textwidth][c]{\Huge Text over Image}}
	\caption{Time series of SPI 3 for the four districts based on the forecast batch of December. Lines in yellow, orange, and brown represent the probability of drought using threshold of mild, moderate, and severe categories, respectively.}\label{fig:ep2}
\end{figure*}


\begin{figure*}
	\noindent
	\resizebox{\textwidth}{\textheight}
	{\includegraphics{../output/prob_plot/mam_jan}}\hspace*{-\textwidth}
	%\raisebox{3in}[0in][0in]{\color{red}
	%\makebox[\textwidth][c]{\Huge Text over Image}}
\caption{Time series of SPI 3 for the four district sbased on the forecast batch of January. Lines in yellow, orange, and brown represent the probability of drought using threshold of mild, moderate, and severe categories, respectively.}\label{fig:ep3}
\end{figure*}


\begin{figure*}
	\noindent
	\resizebox{\textwidth}{\textheight}
	{\includegraphics{../output/prob_plot/mam_feb}}\hspace*{-\textwidth}
	%\raisebox{3in}[0in][0in]{\color{red}
	%\makebox[\textwidth][c]{\Huge Text over Image}}
\caption{Time series of SPI 3 for the four districts based on the forecast batch of February. Lines in yellow, orange, and brown represent the probability of drought using threshold of mild, moderate, and severe categories, respectively.}\label{fig:ep4}
\end{figure*}

\begin{figure*}
	\noindent
	\resizebox{\textwidth}{\textheight}
	{\includegraphics{../output/prob_plot/jjas_mar}}\hspace*{-\textwidth}
	%\raisebox{3in}[0in][0in]{\color{red}
	%\makebox[\textwidth][c]{\Huge Text over Image}}
\caption{Time series of SPI 4 for the four districts based on the forecast batch of March. Lines in yellow, orange, and brown represent the probability of drought using threshold of mild, moderate, and severe categories, respectively.}\label{fig:ep5}
\end{figure*}

\begin{figure*}
	\noindent
	\resizebox{\textwidth}{\textheight}
	{\includegraphics{../output/prob_plot/jjas_apr}}\hspace*{-\textwidth}
	%\raisebox{3in}[0in][0in]{\color{red}
	%\makebox[\textwidth][c]{\Huge Text over Image}}
	\caption{Time series of SPI 4 for the four districts based on the forecast batch of April. Lines in yellow, orange, and brown represent the probability of drought using threshold of mild, moderate, and severe categories, respectively.}\label{fig:ep6}
\end{figure*}

\begin{figure*}
	\noindent
	\resizebox{\textwidth}{\textheight}
	{\includegraphics{../output/prob_plot/jjas_may}}\hspace*{-\textwidth}
	%\raisebox{3in}[0in][0in]{\color{red}
	%\makebox[\textwidth][c]{\Huge Text over Image}}
	\caption{Time series of SPI 4 for the four districts based on the forecast batch of May. Lines in yellow, orange, and brown represent the probability of drought using threshold of mild, moderate, and severe categories, respectively.}\label{fig:ep7}
\end{figure*}

\begin{figure*}
	\noindent
	\resizebox{\textwidth}{\textheight}
	{\includegraphics{../output/prob_plot/mamjja_feb}}\hspace*{-\textwidth}
	%\raisebox{3in}[0in][0in]{\color{red}
	%\makebox[\textwidth][c]{\Huge Text over Image}}
	\caption{Time series of SPI 6 for the four districts based on the forecast batch of February. Lines in yellow, orange, and brown represent the probability of drought using threshold of mild, moderate, and severe categories, respectively.}\label{fig:ep8}
\end{figure*}

\begin{figure*}
	\noindent
	\resizebox{\textwidth}{\textheight}
	{\includegraphics{../output/prob_plot/amjjas_mar}}\hspace*{-\textwidth}
	%\raisebox{3in}[0in][0in]{\color{red}
	%\makebox[\textwidth][c]{\Huge Text over Image}}
\caption{Time series of SPI 6 for the four districts based on the forecast batch of March. Lines in yellow, orange, and brown represent the probability of drought using threshold of mild, moderate, and severe categories, respectively.}\label{fig:ep9}
\end{figure*}